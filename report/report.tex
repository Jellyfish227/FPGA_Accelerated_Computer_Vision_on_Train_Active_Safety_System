\documentclass[12pt, a4paper]{article}
    
\usepackage{homework}
\usepackage{amsmath}				% For Math
\usepackage{graphicx}				% For including figure/image
\usepackage{cancel}					% To use the slash to cancel out stuff in work
\usepackage{multirow}
\usepackage{lastpage}

\usepackage[
backend=biber,
% style=alphabetic,
sorting=ynt,
date=iso,
urldate=iso
]{biblatex}
\addbibresource{refs.bib}

\usepackage[firstpage=true]{background}
\backgroundsetup{
    scale=0.3,
    angle=0,
    opacity=0.1,
    contents={%
        \includegraphics[scale=1]{figs/Emblem_of_CU.png}
    }
}

%%%%%%%%%%%%%%%%%%%%%%
% Set up fancy header/footer
\pagestyle{fancy}
\setlength{\headheight}{28pt}
\fancyhead[LO,L]{CENG3430 Rapid Prototyping of Digital System\\Author: C.H. Yu, C.H. Sennett Cheng}
\fancyhead[CO,C]{}
\fancyhead[RO,R]{Final Project\\Date: \today}
\fancyfoot[LO,L]{}
\fancyfoot[CO,C]{}
% \fancyfoot[RO,R]{Page \thepage\ of \pageref{LastPage}}
\renewcommand{\headrulewidth}{0.4pt}
\renewcommand{\footrulewidth}{0.4pt}

\setcounter{secnumdepth}{3}
\setcounter{tocdepth}{3}
%%%%%%%%%%%%%%%%%%%%%%

\begin{document}  

\begin{titlepage}
    \begin{center}

		\bf\LARGE{The Chinese University of Hong Kong}
        \bf\Large{Department of Computer Science\\and Engineering}
		
		\vspace{80pt}
		
		\vspace{15pt}
		\textbf{\Large CENG3430 Rapid Prototyping of Digital System\\}
		\vspace{10pt}
		\textbf{\Large Final Project - FPGA-based Computer Vision Accelerator for Train Active Safety System}\\
		\vspace{10pt}
		\textbf{\Large Report}\\
        \vspace{6pt}
        % {\large Project Demo Date: 16 May 2025}\\
        {\large Project Demo Deadline: 16 May 2025 23:59}
		\vspace{40pt}
		
        \vspace{15pt}
		\textbf{\normalsize
            1155193237 - Yu Ching Hei (chyu2@cse.cuhk.edu.hk)\\
            1155206044 - Cheng Chung Hei Sennett(1155206044@link.cuhk.edu.hk)\\
        }
		\vspace{40pt}
        \textbf{\large Source code and project history is available on GitHub:}\\
        \normalsize \url{https://github.com/Jellyfish227/FPGA_Accelerated_Computer_Vision_on_Train_Active_Safety_System.git}\\
        \vspace{60pt}
		\textit{Under the kind support and guidance of}\\
		\textbf{\large Prof. Ming-Chang YANG}\\
		\vspace{20pt}
        % \textit{Also thank you for the help provided by teaching assistants}\\
		% \textbf{\large 
        %     Mr. Kezhi LI\\
        %     Mr. Han ZHAO\\
        %     Mr. Zhirui ZHANG\\
        % }
		
	\end{center}
\end{titlepage}

\pagenumbering{roman}
\setcounter{page}{2}
\fancyfoot[RO,R]{Page \thepage}

\tableofcontents

\section*{Division of Work}
\begin{tabularx}{\textwidth}{|l||X|}
    \hline
    \textbf{Task} & \textbf{Person In Charge} \\
    \hline\hline
    \textbf{Project Proposal} & Cheng Chung Hei Sennett \\
    \hline
    \textbf{Data Pre-processing} & Yu Ching Hei, Cheng Chung Hei Sennett \\
    \hline
    \textbf{AI-Inferencing} & Yu Ching Hei \\
    \hline
    \textbf{Post-processing} & Cheng Chung Hei Sennett \\
    \hline
    \textbf{Report} & Yu Ching Hei \\
    \hline
\end{tabularx}

\newpage
\pagenumbering{arabic}
\setcounter{page}{1}
\fancyfoot[RO,R]{Page \thepage\ of \pageref{LastPage}}

\section{Introduction}
What is the system you want to design? Why?
detailed account of the project objectives
and achievements (for future reference, evaluation, and even
reproduction)

The CENG3430 final project involved implementing a computer vision system on the Zynq UltraScale+ MPSoC \cite{zynq_ultrascale_swdev}
that uses the MPU6050 to direct the laser. Our goal was to create a responsive turret for maximizing hits.

\noindent The workflow consists of two phases:
\begin{enumerate}
    \item \textbf{Data Pre-processing:} We collect and process gyro and accelerometer 
          data from the \\MPU6050, then transmit the calculated angles via UART.
    \item \textbf{Data AI-Inferencing:} The angles are received and converted into PWM 
          signals to control the servo motors using Vitis AI \cite{vitis_ai_docs} for efficient inference.
    \item \textbf{Data Post-processing:} The PWM signals are received and converted into 
          the corresponding angles to control the servo motors.
\end{enumerate}
This report outlines our challenges faced and insights gained during implementation, aiming to deliver a functional motion-controlled laser turret.

\section{Design}
Overview: Describe the overall architecture, inputs, and outputs of the
system.
You are highly suggested to use flowcharts and block diagrams for
better presentation.
Module Descriptions: Discuss each module of the system clearly and
the interactions among them.

\subsection{Transmission of Data}
\subsubsection{v1 of data sending formatting(float string)}\text{}
\begin{code}[language=c]
/* Commit hash: c6e8ed6, main_mpu.c*/
void sendData(float yawAngle, float pitchAngle) {
    char data[22];
    // format the data as a string
    sprintf(data, "%.10f,%.10f", yawAngle, pitchAngle);
    char* chp = data;
    while (*chp) 
        UARTCharPut(UART5_BASE, *chp++);
}
\end{code}\text{}\\
In this initial implementation, we formatted the yaw and pitch angles into a string using sprintf. 
While this allowed us to send the angles with high precision (10 decimal places), 
the long string to be sent over UART wirelessly increases the risk of data loss. And turns out during the testing, 
we found that the data received was very unstable, we have presumed that the problem is due to the data being sent is too long, 
which means there are more characters that can potentially be lost, and just one character lost can cause the whole data to be corrupted.

\subsubsection{v2 of data sending formatting(int string)}\text{}
\begin{code}[language=c] 
/* Commit hash: af09000, main_mpu.c */
void sendData(float yawAngle, float pitchAngle) {
    char data[7]; 
    sprintf(data, "%03d%03d\0", (int)yawAngle, (int)pitchAngle);
    /* same send looping as before*/
    UARTCharPut(UART5_BASE, '\n'); // '\n' to mark end of transmission
}
\end{code}
\textbf{}\\
In this update, we have made 2 important changes:
\begin{enumerate}
    \item We have changed the data string to be a string of 7 characters, which is a lot shorter than the previous 22 characters.
    \item We have added a newline character to indicate the end of the transmission, which is a simple way to check the data integrity.
\end{enumerate}
We decided to sacrifice the precision of the data to just send the integral part of the angle, 
in order to reduce the risk of data corruption due to data loss during transmission. 
And in the previous implementation, we found that we have no way to check the data integrity, 
so we have to add a newline character to indicate the end of the transmission.
However, the problem was not fixed, in the slave end, we found that the data received was still very unstable, 
we noticed that sometimes the data received would swap their positions(i.e. the yaw angle would be the pitch angle and vice versa), 
or sometimes some digits that should be in the pitch angle would appears in the array of yaw angle characters received.
Therefore, we started to investigate in the slave end.

\section{Implementation}
Discuss in detail how you implement the system, you can take
screenshots to provide step-by-step instructions, or explain the source
code(s) part by part. You can also provide the screenshot of the
validated block design (if any).

\section{Results}
What have you achieved? Have you realized the preset goals?
What are the difficulties or limitations encountered during the
implementation, and how you resolved them?
What are the further improvement and possibilities?

\section{Conclusion}
To sum up, we have successfully implemented a responsive laser turret that can capture the motion of hand and reflect it to the laser turret wirelessly.
\newline
Before working on this project, we have expected that the most challenging part would be the motion capturing part. 
But surprisingly, we found that the most challenging part was actually the data transmission part, which we have to ensure the data integrity and stability.
Through this iterative process of design, implementation, and debugging, we have gained valuable insights into embedded systems development. 
These experiences and lessons learned will prove invaluable for future embedded systems projects encountered.

\section{References}
\printbibliography[heading=none]

\end{document}